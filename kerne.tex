% coding:utf-8

%----------------------------------------
%FOSAPMA, a LaTeX-Code for a summary of modern mathematics and physics in application.
%Copyright (C) 2013, Mario Felder

%This program is free software; you can redistribute it and/or
%modify it under the terms of the GNU General Public License
%as published by the Free Software Foundation; either version 2
%of the License, or (at your option) any later version.

%This program is distributed in the hope that it will be useful,
%but WITHOUT ANY WARRANTY; without even the implied warranty of
%MERCHANTABILITY or FITNESS FOR A PARTICULAR PURPOSE.  See the
%GNU General Public License for more details.
%----------------------------------------

\chapter{Kerne und Teilchen}
 
\section{Coulomb Gesetz}
Coulomb-Kraft:
\[\boxed{
	F_{Coulomb} = \frac{1}{4 \pi \varepsilon_0} \frac{Q_1 \cdot Q_2}{r^2}
}\]

Coulomb-Energie:
\[\boxed{
	E_{pot} = \frac{1}{4 \pi \varepsilon_0} \frac{Q_1 \cdot Q_2}{r}
}\]
\begin{footnotesize}
	$\varepsilon_0 = 8.8542 \times 10^{-12} \ \ [\frac{C}{V \cdot m}]$
\end{footnotesize}
\\


\section{Atomare Masseneinheit $u$}

\[
	\boxed{
	u = 1.660538921 \times 10^{-27}kg
	}
\]
\\
\begin{footnotesize}
Masse der Elementarteilchen:
\[
\begin{aligned}
	m_p&=1.007276466812u \\
	m_n&=1.008664916u \\
	m_e&=5.4857990946 \times 10^{-4}u
\end{aligned}
\]
\end{footnotesize}


\section{Kerngr�sse}

\[
\boxed{
	R = R_0 \cdot A^\frac13
}
\]
\begin{footnotesize}
$R_0=1.2 \times 10^{-15}m$\\
$A:$ Gesamtzahl Nukleonen, $A=Z+N$\\
\end{footnotesize}


\section{Mol}

1 mol enstpricht der \textbf{Avogadrozahl}:
\[\boxed{
	N_A=6.0221 \times 10^{23}
}\]
\\
Anzahl $mol$ eines K�rpers:
\[\boxed{
	n = \frac{N}{N_A}
}\]
\begin{footnotesize}
	$N:$ Anzahlt Teilchen in einem K�rper
\end{footnotesize}
\\
\\
Masse $m$ eines K�rpers:
\[\boxed{
	m = n \cdot m_{mol}
}\]


\section{Bindungsenergie der Kerne}

Bindungsenergie eines Atomkerns:
\[\boxed{
	E_B=\left(Z \cdot M_H + N \cdot m_n - ^A_ZM \right) \cdot c^2
}\]
\begin{footnotesize}
	$M_H:$ Masse des Wasserstoffatoms ($1.007825u$)\\
	$^A_ZM:$ Masse des neutralen Atoms\\
\end{footnotesize}
\\
mit:
\[\boxed{
	1u \cdot c^2 = 931.494061MeV
}\]
\\
Elementarladung:
\[
	1e=1.602176487 \times 10^{-19}C
\]
\\
\\
Die Atommasse pro Nukleon $\left(\frac{^A_ZM}{A}\right)$ ist ein Mass f�r die Stabilit�t des Kerns.
Je kleiner, desto st�rker die Bindung!
\[\boxed{
	\frac{-E_B}{A}=\frac{\left(Z \cdot M_H + N \cdot m_n - ^A_ZM \right) \cdot c^2}{A}
}\]


\section{Alpha Zerfall}
\begin{figure}[h!]
	\begin{center}
		\leavevmode
		\includegraphics[scale=0.13, trim = 0mm 0mm 0mm 0mm]{images/alpha_decay.png}
        \caption{Alpha Zerfall}
        \label{fig:adecay}
	\end{center}
\end{figure}
~\\
Das Alpha Teilchen ist ein $^4_2He$ Kern.\\
\\
Ein Element $X$ zerf�llt:
\[\boxed{
	^A_ZX \  \rightarrow \ ^{A-4}_{Z-2}Y+\alpha+2e^-
}\]
\\
Atomare Masse von $\alpha$:
\[
	\alpha = 4.001506179125u
\]


\section{Beta Zerfall}
\begin{figure}[h!]
	\begin{center}
		\leavevmode
		\includegraphics[scale=0.13, trim = 0mm 0mm 0mm 0mm]{images/beta-minus_decay.png}
        \caption{Beta-Minus Zerfall}
        \label{fig:bdecay}
	\end{center}
\end{figure}
~\\
Bei einem \textbf{Beta-Minus} Zerfall wird ein Neutron in ein Proton umgewandelt, unter Aussendung eines Antineutrinos.
\[\boxed{
	n \  \rightarrow \  p + \beta^- + \overline{v_e}
}\]
\[\boxed{
	^A_ZX \  \rightarrow \  ^A_{Z+1}Y + \beta^- + \overline{v_e}
}\]
\\
Bei einem \textbf{Beta-Plus} Zerfall wird ein Neutron in ein Proton umgewandelt, unter Aussendung eines Neutrinos.
\[\boxed{
	p \  \rightarrow \  n + \beta^+ + v_e
}\]
\[\boxed{
	^A_ZX \  \rightarrow \  ^A_{Z-1}Y + \beta^- + \overline{v_e}
}\]


\section{Gamma Zerfall}
\begin{figure}[h!]
	\begin{center}
		\leavevmode
		\includegraphics[scale=0.13, trim = 0mm 0mm 0mm 0mm]{images/gamma_decay.png}
        \caption{Beta-Minus Zerfall}
        \label{fig:bdecay}
	\end{center}
\end{figure}
~\\
Beim Gamma Zerfall befindet sich das Atom in einem angeregten Zustand und sendet dan ein hochenergetisches Photon ($\gamma$) aus.\\
\\
Es ensteht kein neues Element, sondern der Kern f�llt nur in einen energetisch tieferen Zustand.\\
\\
\[\boxed{
	E_{Photon}=h \cdot f
}\]
\\
Plancksche Konstante:
\[
	h=6.6261 \times 10^{-34}J \cdot s
\]


\section{Elektromangetisches Spektrum}
F�r jede elektromagnetische Welle gelten folgende Gleichungen:
\[\boxed{
\begin{aligned}
	c &= f \cdot \lambda \\
	p_\gamma &= \frac{h}{\lambda} \\
	E_\gamma &= h \cdot f = h \cdot \frac{c}{\lambda} = p_\gamma \cdot c
\end{aligned}
}\]
\begin{footnotesize}
	$f:$ Frequenz \\
	$\lambda:$ Wellenl�nge \\
	$p_\gamma:$ Impuls \\
	$E_\gamma:$ Energie des Photons\\
	\\
	$c = 299792458 \frac{m}{s}$\\
	$h = 6.626069 \times 10^{-34} J \cdot s$
\end{footnotesize}


\section{Halbwertszeit und Aktivit�t}
Radioaktives Zerfallsgesetz:
\[\boxed{
	N(t)=N_0 \cdot \ex{-\beta \cdot t}
}\]
\begin{footnotesize}
	$\beta:$ Zerfallskonstante
\end{footnotesize}
\\
\\
Halbwertszeit:
\[\boxed{
	T_{\frac12} = \frac{\ln{2}}{\beta} = \ln{2} \cdot \tau
}\]
\begin{footnotesize}
	$\tau:$ Zerfallszeit ($\frac1\beta$)
\end{footnotesize}
\\
\\
Aktivit�t:
\[\boxed{
	A=\left| \frac{\di N}{\di t} \right| = \beta \cdot N = \frac{N}{\tau}
}\]
\begin{footnotesize}
	Becquerel: $1 Bq = 1$ Zerfall/Sekunde\\
	Curie: $1 Ci = 3.70 \times 10^{10}Bq$
\end{footnotesize}


\section{$^{14}C$-Datierung}
Alles Lebendige enth�lt $^{14}C$. Das $^{14}C$ entsteht kontinuierlich durch kosmische Bestrahlung.\\
\\
Das nat�rliche Verh�ltnis:
\[\boxed{
	\frac{^{14}C}{^{12}C} = 1.3 \times 10^{-12}
}\]
\\
\\
Halbwertszeit von $^{14}C$:
\[\boxed{
	T_{\frac12} = 5730 \mathrm{\text{ Jahre}}
}\]
\\


\section{Strahlendosis}
Die absorbierte Energie pro kg lebendes Gewege nennt man \textbf{Energiedosis} D:
\[\boxed{
	[D] = 1 Gray = 1Gy = 1 \frac{J}{kg} = 100rad
}\]
\\
Der biologische Effekt wird beschrieben durch die \textbf{�quivalentdosis} H:
\[\boxed{
	[H] = 1 Sievert = 1 Sv = 100rem
}\]
\\
Bewertugnsfaktor q:
\[\boxed{
	H = D \cdot q
}\]
\begin{footnotesize}
	$[q] = \frac{Sv}{Gy}$
\end{footnotesize}
\\
\\
\begin{tabular}{|l|c|}
	\hline
	Strahlung		&	$q$ [$\frac{Sv}{Gy}$]	\\
	\hline
	\rowcolor{white}$\gamma$		&	1				\\
	\rowcolor{lgray}$\beta$		&	1 - 1.5			\\
	\rowcolor{white}langsame $n$	&	3				\\
	\rowcolor{lgray}$n$ (0.02 - 0.1MeV)&	5 - 8				\\
	\rowcolor{white}schnelle $n$ und $p$&10				\\
	\rowcolor{lgray}$\alpha$		&	20				\\
	\rowcolor{white}schwere Kerne 	&	20				\\
	\hline
\end{tabular}\\
\\
\\
Die Frei werdende Energie einer Nuklearexplosin wird mit dem Energie�quivalent des Sprengstoffs TNT verglichen.
\[\boxed{
	1kT = 4.184 \times 10^{12}
}\]
\\


\section{Wenig Zerf�lle - Poisson Statistik}
Der radioaktive Zerfall eines Kerns ist \textbf{echt zuf�llig}
\[\boxed{
	p(x) = \frac{m^x \cdot \ex{-m}}{x!} = \frac{m^x \cdot ex{-m}}{\int\limits_0^\infty \left(t^x \cdot \ex{-t} \right) \di t}
}\]
\begin{footnotesize}
	$p(x)$ ist die Wahrscheinlichkeitsdichte f�r den Wert $x$ bei einem Mittelwert von $m$.
\end{footnotesize}
\\


\section{Viele Zerf�lle - Gauss Statistik}
F�r lange Messzeiten, d.h. gross Mittelwerte geht die Poisson Verteilung in die Gaussverteilung �ber:
\[\boxed{
	p(x) = \frac1{\sqrt{2\pi \cdot m}} \cdot \ex{\frac{-\left(x - m\right)^2}{2m}}
}\]
\\
Standard Abweichung:
\[
	\sigma = \sqrt{m}
\]
\\


\section{Nukleare Reationen}
Zwei Kerne $A$ und $B$ verwandeln sich in einem Prozess zu Kernen $C$ und $D$.
Die \textbf{Reaktionsenergie} Q berechnet sich:
\[\boxed{
	Q = \left( M_A + M_B -M_C -M_D \right) \cdot c^2
}\]


\subsection{exotherm}
Ist Q positiv, wird die gesamte Masse kleiner und die kinetische Restenergie gr�sser.

\subsection{endotherm}
Ist Q negativ, wird die gesamte Masse gr��ser und die kinetische Energie kleiner.\\
Eine endotherme Reaktion ist nur m�glich, wenn die gewonnene Masse in Form kinetischer Energie vorhanden ist.
Dabei entspricht der Betrag von Q gleich der kinetischen Energie im Schwepunktsystem:
\[
	\left|Q\right| = Q_{cm}
\]
\\
Wenn der einte K�rper in Ruhe ist, muss sich der andere mit folgender Energie ann�hern:
\[\boxed{
	E_{kin,a}=\frac{M_b + M_a}{M_b} \cdot Q_{cm}
}\]


\section{Aktivierungsenergie}
kinetische Energie von Teilchen bei Temperatur T:
\[
	E_{kin} = \frac{3}{2} k_B \cdot T
\]
\begin{footnotesize}
	$k_B = 1.3806488 \times 10^{-23}\ \frac{J}{K}$
\end{footnotesize}